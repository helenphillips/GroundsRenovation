\ifappendixStyle %% thesis is true, journal is false
\section{Introduction} %% Thesis
\else
\section*{Introduction}
\fi


Urbanisation has increased globally and this increase is set to continue \citep{Heilig:2012un}. Urban expansion has resulted in the widespread loss of natural and semi-natural habitats, both directly and indirectly, which are important as refuges and corridors for biodiversity \citep{Goulson:2002oe,Osborne:2008jpe} and for human well-being \citep{Fuller:2007bl,Dallimer:2012bs,Bratman:2015nature,Shanahan:2016sr}. Any retention or creation of green-spaces within urban areas is therefore considered important (\citealt{Alvey:2006uf}, but see \citealt{Dearborn:2010cb}).

Although urban ecology has been a neglected facet of ecology, the field's popularity has been growing \citep{Mcphearson:2016Bs}. Several UK-based projects have assessed urban biodiversity \citep{Gaston:2004bugs,Angold:2006ste}, investigating how it can be maintained and improved. It is important that these findings are communicated, especially to the general public, so that decisions regarding urban gardens can result in enhanced biodiversity potential \citep[e.g.][]{Thompson:2007no}.

The Natural History Museum in London (NHM) has 2.18 hectares of grounds around its buildings at its main South Kensington site, which for ease of reference can be split (at the centre of the museum building) into the ``eastern area'' and ``western area'' (Figure \ref{fig:wlgplan_after}a). The museum grounds were renovated in 1995 with the creation of a one-acre (0.4 hectares) Wildlife Garden (henceforth WLG; \citealt{Honey:1999ln}) in the western area, which contains small areas of multiple lowland habitats present in southern England. The eastern area is heavily and repeatedly disturbed due to temporary attractions (a butterfly exhibit in the summer and ice rink in the winter); at other times, it contains only regularly-replaced amenity grassland and areas of introduced shrubs with no habitats traditionally considered ``wildlife-friendly''. 


\begin{figure}
	\centering

	\begin{subfigure}[t]{0.45\textwidth}
		\centering
		\includegraphics[scale=0.3, angle =90, trim={0 3cm 0 2cm},clip]{BiodiversityStrategyExistingSite}
       		 \caption{}\label{fig:fig_a}
	\end{subfigure}
%
	\begin{subfigure}[t]{0.45\textwidth}
		\centering
		\includegraphics[scale=0.3, angle =90, trim={0 3cm 0 2cm},clip]{BiodiversityStrategyProposedSite}
		\caption{}\label{fig:fig_b}
	\end{subfigure}
	\caption{Detailed plans of the NHM grounds, provided by Wilder Associates, and the area (m\textsuperscript{2}) of: (a) Current habitat types and, (b) Proposed habitat types.}
	\label{fig:wlgplan_after}
\end{figure}

Since free admission to the Museum was introduced in 2001, it has had an average of around 5 million visitors per year, with this number expected to increase. In part to alleviate the pressure of such large visitor numbers on the two current entrances, a third entrance through the Darwin Centre, at the west of the building, has been proposed. A Supplementary Planning Document from the Royal Borough of Kensington and Chelsea \citep{rbkc:2012} required the Museum to develop a unified scheme across the grounds to balance the competing demands from historical significance of the iconic Waterhouse building, the setting of the listed building, visitor amenity and use of space for events. To this end, an international design competition was held in Autumn 2013. The competition attracted 43 entries from around the world, reduced to six finalists in early 2014 and a single winner unanimously chosen by a jury, chaired by Ian Henderson CBE, in April 2014. The competition was won by the team Niall McLaughlin Architects and Kim Wilkie, whose plan contained an overarching continuous theme similar to that of the museum building, moving from ``extinct'' habitats in the east to current British habitats in the west (Figure \ref{fig:wlgplan_after}b, also see http://www.kimwilkie.com/london/natural-history-museum). The plans as proposed will result in the loss and reduction of some habitats, gain and expansion of others, and disturbance particularly in the eastern part of the grounds. The proposed changes prompted concerns for the wildlife currently inhabiting the grounds, especially in the WLG in the southwestern corner of the site \citep{prospect:2015wg,guardian:2015wg,avery:2015wg,telegraph:2015wg,dailymail:2015wg}: a petition to stop the redevelopment of the grounds attracted over 37,000 signatures as of 1 June 2016 \citep{changepetition:2015wg}.  

Among the arguments used by critics of the proposals is that the grounds harbour unusually high levels of biodiversity \citep{changepetition:2015wg}, which would be jeopardised by the proposed changes to the grounds \citep{avery:2015wg}. Over 2800 species have been recorded from the WLG in the 21 years since its creation, in occasional structured surveys and more haphazard observations \citep{Ware:2016wlg}. Despite species' having been recorded since 1995 when the WLG was created, new species continue to be added to the cumulative list of taxa recorded. Perhaps the most notable was that of the earthworm \textit{Dendrobaena pygmaea}, not recorded in the UK for 32 years prior to its discovery in the WLG. Some previously unrecorded species have probably been able to colonise because of disturbance within and at the edge of the WLG \citep{Ware:2016wlg}. For example, common cudweed (\textit{Filago vulgaris}) and flixweed (\textit{Descurainia sophia}), both vascular plants rare in London, were able to grow following disturbance from the building of the Darwin Centre in 2010 \citep{Ware:2016wlg}. Species accumulation curves may be beginning to plateau in some taxa. For example, between 1995 and 2003, 100 species of algae were recorded, but only ten further species have been added since 2003. However, other taxa show no sign of saturation: 18 species of Araneae were recorded by 2003, for instance, but this had risen to 80 by 2015 \citep{Ware:2016wlg}.

However, lengths of lists of recorded species can only be compared meaningfully across sites if sampling effort has itself been recorded or, better yet, been equal at each site; otherwise, lengths of lists typically conflate differences in sampling effort with true diversity differences \citep{Gotelli:2001el,Crawley:2005flora}. For example, assiduous sampling led to 2204 species of plant and animals from selected groups being recorded over 15 years from a domestic garden in Leicester \citep{Owen:1991ecology}. Because species in many high-diversity taxonomic groups, such as insects or other invertebrates, can often be differentiated only by taxon specialists, taxonomic expertise can also influence lengths of species lists \citep{Crawley:2005flora}. The Natural History Museum provides one of the greatest concentrations of such expertise in the world, meaning the list of species from the WLG is likely to be more comprehensive than those from almost anywhere else on earth.

An independent ecological assessment of the biodiversity value of the WLG, done as part of a planning application, suggested that, apart from breeding birds, a number of invertebrates and the accidental introduction of a slow worm, ``No other protected or noteworthy species were considered likely to be supported within the site'' \citep{ImpactAssessment:2015cf,PrelimEcoAppraisal:2015cf}, although some protected species had been seen foraging in the garden (common and soprano pipistrelle bats). These findings are in line with expectations for young anthropogenic habitat patches in an urban setting. However, as with the lists of recorded species, this assessment did not provide any quantitative estimates of diversity that could provide the basis of a comparison between the biodiversity of the current grounds and that expected or (in future) found under the new proposal.

Few robust tools are available to estimate potential impacts to biodiversity from development and land-use change, especially at such small spatial scales. For planning applications it is advised, although not always a necessity, that ecological surveys (desk-based or field-based surveys as part of a Preliminary Ecological Assessment and/or an Ecological Impact Assessment) be conducted prior to submission to determine, amongst other things, how species and habitats at the site might be impacted by the proposed works \citep{Cieem:2016}. However, especially in the case of a desk-based survey, these methods would be unable to estimate the likely gains or losses of biodiversity until after the fact. DEFRA's Biodiversity Offsetting model \citep{defra:2012bdo} offers a potential way of assessing potential impact on biodiversity via the habitat types that are to be displaced, in order to prevent net loss of biodiversity. Briefly, each habitat type carries a distinctiveness score (2, 4 or 6); each patch is assigned a condition score (1, 2 or 3), and these are multiplied together to calculate a per-hectare biodiversity score; multiplying this by the habitat patch area and summing the result across all patches gives an overall biodiversity score. For increased or new areas of biodiversity-rich habitat, scores are moderated to reflect the time needed to achieve the target level of biodiversity and the risk that it will never be reached. The score of the proposed habitat types would need to match or exceed the score of the habitats being displaced. Although operational, this offsetting method falls short in urban environments (habitats are presumed to be in a natural setting), and the scores are not strongly grounded in relevant biodiversity data \citep[see][]{Baker:2014bdo}.  

A common approach in conservation ecology to estimating the effects of land-use change on biodiversity is to undertake comparable ecological surveys at nearby sites in different land uses, under the assumption that such spatial comparisons can be used in lieu of time-series data tracking biodiversity through land-use changes. Although no such data have been published from within the WLG itself, such comparisons are sufficiently common to permit powerful global syntheses \citep[e.g.][]{Alkemade:2009globio3,Gibson:2011fk,Gerstner:2014jae}. In particular, the PREDICTS project has modelled data from surveys worldwide to estimate how land-use change and related pressures affect occurrences and abundances of many species \citep{Newbold:2014procb,DePalma:2015jae} and broader site-level measures of biodiversity \citep{Newbold:2015nat,Newbold:2016sci,DePalma:2016scirep}. By focusing on surveys that have included sites in different land uses, this approach is able to estimate relative levels of biodiversity for each land use type (even if no single survey represents the full range of land uses). Because biodiversity is linked to pressure data in a dose-response modelling framework, the model can be combined with projections of future land use and other pressures to estimate average levels of site-level biodiversity in the future, enabling comparison with the present \citep{Newbold:2015nat}. The PREDICTS framework is therefore designed to tackle the similar kinds of question to those posed by the grounds redevelopment, such as, will the development cause a negative effect on biodiversity over the long term?

Given this conceptual similarity, aware of the controversy surrounding the biodiversity costs and benefits of the proposed development, and having no involvement in either the proposal or the opposition to it, two of us (HRPP and AP) offered to undertake an analysis for the Natural History Museum, conceptually derived from that of \cite{Newbold:2015nat}, to estimate the net effects of the proposal and to make the resulting estimate public. The proposal was accepted by the Natural History Museum, on a short three-month timescale. SK, already involved in the Grounds Transformation Project, joined the analysis and provided detailed information about the current and proposed layouts of the grounds, as well as facilitating access to the dataset of species recorded from the various habitats within the WLG.

We extended the analytical framework developed by \cite{Newbold:2015nat} to allow for the fact that the spatial extent of a habitat, as well as its type, is likely to affect its biodiversity values. Larger habitat patches are expected to contain not only more species overall than smaller patches (in line with the species-area relationship: e.g. \citealt{rosenzweig:1995species}), but also -- though less strongly -- more species per unit area (i.e., greater species density; 
\ifappendixStyle %% thesis is true, journal is false
Chapter \ref{ch:sdar})%% Thesis
\else
\citealt{Phillips:2015sdar})
\fi
. Our analysis aims to take such area-dependency into account. Although many other factors can also affect site-level diversity, notably habitat age \citep{Sattler:2010le}, time limitations precluded their consideration; a point to which we return in the discussion.



%%% Table 1
\afterpage{
%\clearpage
\begin{landscape}
\centering
\scriptsize
\renewcommand*{\arraystretch}{1.4}
\begin{longtable}{B{2.4cm}  B{8cm} B{1.4cm} B{1.2cm} B{1.2cm} B{4.4cm}}
\caption{Habitat types in the current and the proposed plans of the Natural History Museum grounds. For each habitat type a brief description is given, its UK BAP Broad Habitat classification, current area and area under the proposed plans and how the coefficient for the biodiversity estimate was obtained.}
\label{tab:wlg1} \\
Habitat Type	&	Description	&	UK BAP Broad Habitat	&	Current Area (m\textsuperscript{2})	&	Proposed Area (m\textsuperscript{2})	&	Coefficient\\ \hline
Hard-standing	&	Pathways and other concreted areas		&	NA	&	10415	&	9525.16	&	Assumed to be zero \\
Amenity grass/turf	&	Gardens, lawns or turfed areas	&	NA	&	3303.63	&	1573.91 &		Modelled\\
Introduced shrubs	&	Beds planted with introduced species, with occasional trees	&	NA	&	2218.62	&	1346.69	&	Broadleaved woodland coefficient adjusted based on Strong et al. 1979\\
Neutral grassland	&	Rotational grazing by sheep during late summer months and autumn. Area estimates include the semi-improved grassland	&	Neutral grassland	&	2103.15	&	2133.45	&	Modelled\\
Broadleaved woodland	&	Mixed tree species, usually dominated by pedunculate oak (Quercus robur) and silver birch (Betula pendula), understory typically comprised of hazel (Corylus avellana) and holly (Ilex aquifolium)	&	Broadleaved, mixed and yew woodland	&	1978.36	&	3477.67	&	Modelled\\
Short/perennial vegetation		&	Ephemeral vegetation, such as common nettle (Urtica dioica), dandelion (Taraxacum officinale agg.) and creeping buttercup (Ranunculus repens)	&	NA		& 423.65	&	0	&	Modelled\\
Chalk grassland	&	Species richness grassland, abundant species include kidney vetch (Anthyllis vulneraria) and sheep's fescue (Festuca ovina) &	Calcareous grassland	&	344.58	&	526	&	Modelled\\
Ponds	&	Currently three ponds (70m\textsuperscript{2}, 90m\textsuperscript{2} and 400m\textsuperscript{2}) with linked water systems. Designed to be typical of chalk and peat ponds, but currently contain similar plant communities. Proposed plans contain two ponds 	& 	Standing water and canals	&	341.28	&	459.37	&	Modelled\\
Marginal vegetation (pond edge)	&	 Pond surrounding, dominated by common reed (Phragmites australis)	&	Standing water and canals	&	163.6	&	99.15	&	Modelled\\
Species-rich hedgerow	&	Hedgerow with more than one native species, typically dominated by hawthorn (Crataegus monogyna)	&	Boundary and linear features	&	121.87	&	607.5	&	Modelled\\
Species-poor hedgerow	&	Single species hedgerow	& 	Boundary and linear features	&	109	&	0	&	Species-rich hedgerow coefficient adjusted based on Scriven et al. 2013.\\
Acid grassland (heath)	&	Included both wet and dry acid grassland	&	Dwarf shrub heath	&	100	&	82	&	Modelled\\
Fen (including reedbed)	&	Fen species included marsh fern (Thelypteris palustris), common reed (Phragmites australis) and lesser pond sedge (Carex acutiformis)	& 	Fen marsh and swamp	&	64.6		&	133.86	&	Modelled\\
Green roof	&	Planting on top of shed	&	NA	&	9.98	&	0	&	Modelled, based on Canadian study\\
Ferns and cycad planting	&	Plantings of (predominantly) non-native ferns and cycads	&	NA	&	0	&	729.82	&	Introduced shrubs coefficient\\
Agricultural plants	&	Rotating crop plantings, species similar to those planted in allotments	&	NA	&	0	&	583.97	&	Modelled\\
Cretaceous Angiosperm shrubs		& Angiosperms similar to those present during the late Cretaceous period		&	NA	&	0	&	244.97	&	Broadleaved woodland coefficient adjusted based on Strong et al. 1979\\
Paleogene Asteraceae	&	Asteraceae simiar to those present during the Paleogene period	&	NA	&	0	&	176.57	&	Short/perennial vegetation coefficient\\
Neogene grass	&	Grass similar to that present during the Neogene period	&	NA	&	0	&	156.27	&	Amenity grass/turf coefficient \\ \hline
      \bottomrule

\end{longtable}

\end{landscape}
%\clearpage
}






\ifappendixStyle %% thesis is true, journal is false
\section{Methods}
\subsection{Study Site}%% Thesis
\else
\section*{Methods}
\subsection*{Study Site}
\fi

There are 1.8 hectares of grounds around the NHM buildings at South Kensington. The entire green space comprising the grounds has been designated a non-statutory Site of Borough Importance for Nature Conservation (SINC) grade II, and is in close proximity to two other non-statutory SINCs; (i) Prince's Gate East, Prince's Gate West and Rutland Gate North, and (ii) Hyde Park and Kensington Gardens. The NHM grounds, both current and post-renovation, were classified into 19 different habitat types, terrestrial and aquatic, some of which can be linked to the UK BAP Broad habitat classes (Table \ref{tab:wlg1}).

\ifappendixStyle %% thesis is true, journal is false
\subsection{Biodiversity measure}%% Thesis
\else
\subsection*{Biodiversity measure}
\fi

Biodiversity is a complex, multifaceted and multiscale concept that cannot be captured fully by any single measure \citep{Purvis:2000nat}. Given time constraints, we therefore had to choose the most appropriate measure of biodiversity to include in our models. Perhaps the most intuitively appealing would be the overall species richness of the grounds. However, as outlined above, the sampling undertaken so far does not provide a basis for estimating this quantity in the present, and even if it did there would be no basis for estimating overall species richness under the proposed changes.

\cite{Newbold:2015nat} focused mainly on within-sample species richness and overall abundance, both expressed relative to the values expected for a pristine site (i.e., a site with no human impacts). Such a baseline is not appropriate for young anthropogenic urban habitats, which are typically not expected to approach the diversity of pristine habitats and which are not in close geographic proximity to any such habitats. Additionally, \cite{Newbold:2015nat} did not consider the effects of habitat patch size on within-sample species richness, despite the expectation of a positive correlation 
\ifappendixStyle %% thesis is true, journal is false
(Chapter \ref{ch:sdar})%% Thesis
\else
(\citealt{Phillips:2015sdar})
\fi
. To overcome these twin limitations, we chose to use a measure of biodiversity that can incorporate effects of patch size -- namely species density (the expected number of species sampled in a constant area of a given habitat; \citealt{Whittaker:2001uz,Magurran:2004tn}) -- and did not attempt to express values relative to a pristine baseline.

\ifappendixStyle %% thesis is true, journal is false
\subsection{Collation of data}%% Thesis
\else
\subsection*{Collation of data}
\fi

We conducted literature searches to identify publications that compared within-sample species richness between two or more of the habitat types in Table \ref{tab:wlg1}. Two searches were undertaken: the first set of search terms was highly specific (full search terms in Appendix \ref{txt:wlg-appendix1}) while the second -- to fill the many remaining gaps -- was broader (full search terms in Appendix \ref{txt:wlg-appendix2}). Additional searches targeted habitats for which data were lacking, particularly habitats which are not typically urban or widespread in the UK.

We used data collected from urban environments wherever possible. Data had to meet four criteria:
\begin{enumerate}
\item The study sampled invertebrates and/or plants in more than one habitat type and/or within a habitat of differing area or age.
\item Sampling was undertaken within the UK (with the exception of samples from green roofs, as no suitable UK data were found).
\item The paper presented the area over which the sampling was conducted; this area was either the sampling frame or the size of the patch of habitat (if the entire patch was sampled). 
\item Data were presented as species richness values, although abundance measures were also recorded if presented.
\end{enumerate}
ImageJ \citep{schindelin:2012fiji} was used to extract data from figures when required. We did not find sufficient data that compared habitats of different ages or that reported measures of abundance, so these aspects of the original design of the study were dropped for practical reasons (the study had to be completed by a deadline in order to be able to feed into the planning process).

The data from each paper were collated as a ``study''. If a paper contained data from multiple sampling methodologies then it was split into multiple studies based on the methodology \citep[following][]{Hudson:2014predicts}. Data were recorded for each site within a study where possible, or otherwise as averages/totals for each habitat type within a study. For each study, we recorded whether it sampled invertebrates or plants. We classified the habitat of each site into one of the 19 habitat types in Table \ref{tab:wlg1}; any sampled habitats not present in the museum's grounds or renovation plans were excluded from the analysis. (The complete dataset is available at: \url{http://data.nhm.ac.uk/dataset/grounds-metaanalysis-data} and code is available at: \url{https://github.com/helenphillips/GroundsRenovation})

\ifappendixStyle %% thesis is true, journal is false
\subsection{WLG Plant Database}%% Thesis
\else
\subsection*{WLG Plant Database}
\fi

Data on plants from the WLG database were also included in the modelling dataset to increase the robustness of some habitat comparisons. The WLG is currently split into 55 zones of different size (see Figure 1 in \citealt{Leigh:2003ln}), with each zone's assemblage originally planted based on National Vegetation Classification communities \citep{Honey:1999ln,Rodwell:1998nvc}. Between 1995 and 2015 a complete inventory of the plant species in each zone has been completed non-systematically every year. Because the database species binomials included some synonyms, species names of all records were standardised using the UK Species Inventory (UKSI) database \citep{Raper:2014wg}. Current WLG habitat types of each zone were taken from \citep{Leigh:2003ln} and confirmed by WLG habitat managers. With advice from members of the Grounds Project team, we classified each zone into a habitat type (Table \ref{tab:wlg1}) and the species richness of each zone was calculated as the total number of species surveyed between 2013 and 2015 (on the grounds that species might be missed in any year, and that more recent surveys are more relevant to the current state). Each zone was treated as a site, with the area estimated through digitisation of figure 1 in \cite{Leigh:2003ln}. Although the WLG database also contains data on other groups of organisms, such as invertebrates, these were not suitable for our analysis as sampling effort and methodology were too heterogeneous.

\ifappendixStyle %% thesis is true, journal is false
\subsection{Accounting for area effects}%% Thesis
\else
\subsection*{Accounting for area effects}
\fi
As well as depending on the nature of the habitat, the expected number of species in a sample also depends on the area covered by the sample (the species-area relationship, or SAR: \citealt{rosenzweig:1995species}) and the extent of the (often much larger) habitat patch within which the sample was taken (the species density-area relationship, or SDAR: 
\ifappendixStyle %% thesis is true, journal is false
Chapter \ref{ch:sdar}). %% Thesis
\else
\citealt{Phillips:2015sdar}). 
\fi
Samples covering larger areas will encounter a wider range of microclimatic and other environmental conditions, meaning that more species have the potential to be sampled. Larger patches of habitat can additionally support larger populations of resident species meaning that species density is likely to be higher. Both of these relationships need to be considered in order to provide the best estimate of the net effects of the proposed redevelopment on biodiversity within the grounds.

We estimated the expected species density for a 10m\textsuperscript{2} sampling frame, from each site's within-sample species richness and area sampled, using: 
\begin{equation}
log S_{10} = log S_s + z(log 10 - log A_s)
\end{equation}


Where $A_s$ is the area over which the sample was taken and $S_s$ the number of species in the sample, and $10$ is the area for which species density ($S_{10}$) is calculated for. Theory predicts that z {\raise.17ex\hbox{$\scriptstyle\sim$}} 0.10 
\ifappendixStyle %% thesis is true, journal is false
(Chapter \ref{ch:sdar})%% Thesis
\else
(\citealt{Phillips:2015sdar})
\fi
: the difference between the island SAR for isolated fragments (z {\raise.17ex\hbox{$\scriptstyle\sim$}} 0.25) and the continental SAR (z {\raise.17ex\hbox{$\scriptstyle\sim$}} 0.15). 
\ifappendixStyle %% thesis is true, journal is false
In Chapter \ref{ch:sdar} I %% Thesis
\else
\cite{Phillips:2015sdar} 
\fi
tested this prediction, estimating z empirically from a synthesis of data from 38 studies; the empirical estimate was z = 0.07, but the predicted value fell within the 95\% confidence interval (0.048 to 0.11). We therefore use z = 0.10 in the analyses that follow, but present results of z = 0.7 in Appendix \ref{txt:wlg-appendix3} as a further sensitivity analysis. Because the area-scaling of species density is not yet well established \citep[e.g.][]{Giladi:2014bio}, we also modelled within-sample species richness as a response variable.

\ifappendixStyle %% thesis is true, journal is false
\subsection{Modelling}%% Thesis
\else
\subsection*{Modelling}
\fi

A generalised linear mixed-effects model was used to estimate average species density (per 10m\textsuperscript{2}) and species richness for each habitat type. Both response variables were rounded to the nearest integer to allowed for the appropriate error structure, as count data are expected to follow a poisson distribution. Study identity was included as an intercept-only random effect to account for differences in methodology and the resulting heterogeneity of the data \citep{Zuur:2009me}. The maximal models included habitat type with an additive effect of taxonomic group as fixed effects; there was not enough data to create a meaningful interaction between the two main effects. The main effects were simplified based on log-likelihood ratios \citep{Zuur:2009me,Crawley:2012r}.

Six habitat types (Table \ref{tab:wlg1}) were not represented by enough data for an average species density or species richness to be modelled. For these six habitats, species diversity (density and richness) was estimated either using the modelled coefficient from another, similar, habitat type; or by using a single study to relate species diversity to the estimated coefficient for another habitat type. The last column of Table \ref{tab:wlg1} gives details of these estimates. Additionally, we assumed that hard standing had zero species richness.

\ifappendixStyle %% thesis is true, journal is false
\subsection{From statistical models to estimates of biodiversity}%% Thesis
\else
\subsection*{From statistical models to estimates of biodiversity}
\fi

For both the current grounds and the proposed redevelopment, we combined the areas of each habitat type with the coefficients of our models in order to estimate overall biodiversity, so that these estimates could be compared to assess the net changes. We explored the effects of three alternative assumptions when using our model coefficients.

\textbf{Assumption 1: Area-scaling of both input data and model output.} For each habitat patch, we used the appropriate coefficient from our model of species density, but rescaled it to the area of the habitat patch to reflect the area-scaling of species density. Scaling species density for habitat area assumes that the habitat is effectively contiguous (i.e., any breaks in the habitat do not prevent movement or dispersal across them). Although this is typically the case in the renovation plans, it is less so in the current grounds. Thus, any bias caused by this assumption will tend to overestimate the overall biodiversity value of the current grounds.

\textbf{Assumption 2: Area-scaling of input data only.} For each habitat patch, we used the appropriate coefficient from our model of species density, but did not rescale it to the area of the habitat patch. 

\textbf{Assumption 3: No area scaling.} For each habitat patch, we used the appropriate coefficient from our model of within-sample species richness. Most comparisons of species richness among habitats do not consider effects of area on the numbers of species sampled at all; we therefore also modelled this possibility.

For each assumption in turn, and for each layout (current or proposed), we computed the area-weighted sum of habitat scores; i.e., each habitat's biodiversity score was multiplied by its area in that layout, and the products summed across all habitat patches. Within each assumption, these scores can be compared between the current and proposed layouts.

\ifappendixStyle %% thesis is true, journal is false
\subsection{Sensitivity Analysis}%% Thesis
\else
\subsection*{Sensitivity Analysis}
\fi

A sensitivity analysis was performed to assess the robustness of the modelled species density coefficients under Assumption 1. For each of the 19 habitats, a coefficient was drawn from a normal distribution whose mean was the estimate of species density (per 10m\textsuperscript{2}) and standard deviation the standard error \citep{Newbold:2015nat}. For the six habitats without modelled coefficients, the means were calculated as above (Table \ref{tab:wlg1}), with standard errors of the same habitat type also being used but multiplied by 1.5 to reflect the increased uncertainty. The total weighted species density values for before and after the grounds renovation were calculated, as above, and the percent change between the two recorded. This process was repeated 1000 times and the frequency of negative change (i.e., biodiversity loss under the proposed plans) determined.

\ifappendixStyle %% thesis is true, journal is false
\subsection{Compositional Similarity}%% Thesis
\else
\subsection*{Compositional Similarity}
\fi

Community similarity of the habitats within the current WLG was estimated using the plant database as an indication of how overall species composition might change with the removal of some habitats. Using just the records in the database between 2013 and 2015, the similarity of species composition (percentage of species in common) was calculated between each pair of habitat types, and the results displayed as an asymmetrical matrix. Thus, for each habitat on the x-axis, the percentage of species that habitat-x shares with a habitat on the y-axis is shown.

\ifappendixStyle %% thesis is true, journal is false
\section{Results}
\subsection{Meta-Analysis}%% Thesis
\else
\section*{Results}
\subsection*{Meta-Analysis}
\fi

The first literature search gained 101 articles and the second literature search acquired 1158 articles. Further targeted searches acquired data from an additional 5 articles. Based on the data criteria, only data presented in 11 papers were suitable for modelling; these were collated into 14 studies based on methodology. These studies contained sampled sites from across the UK (Figure \ref{fig:wlgmap}), as well as one study in Canada, and included suitable data we were able to access from the WLG database.

\begin{figure}[t]
	\centering
	\includegraphics[scale=0.5]{Map_ModelledStudes}
	\caption{Map of the 12 U.K. studies (10 papers) included in the analysis (Data Sources: \cite{Petit:1998bc,Wilson:2003aee,Fountain:2004et,Smith:2006ue,Butt:2008ejsb,Williams:2008hb,Scriven:2013ije,Sirohi:2015jic,Speak:2015ufug}, WLG Database). \cite{Macivor:2011ue} was included in the analysis (containing two studies) but is not shown on this map.}
   	 \label{fig:wlgmap}
\end{figure}

The fixed effects of the mixed-effects model with species density (per 10m\textsuperscript{2}) as the response variable were simplified by the removal of the additive effect of taxon. Species density (per 10m\textsuperscript{2}) significantly varied between habitat types ($\chi^2$ = 353.18, d.f. = 12, p < 0.01; Figure \ref{fig:wlgmodel}). Chalk grassland had the highest species density (per 10m\textsuperscript{2}), whilst pond and fen had the lowest among habitats for which sample-based data were available. 

Similar to the species density model, the species richness model was also simplified with the removal of the additive effect of taxon. Species richness significantly varied with habitat types ($\chi^2$ = 468.01, d.f. = 12, p < 0.01; Figure \ref{fig:wlgmodel}). The relative diversity of each of the habitats was largely consistent among the two models.
 	 
\begin{figure}[t]
	\centering
	\includegraphics[scale=0.5]{Habitat_density&Richness_plusmissing}
	\caption{Model estimates of the 19 habitats within the Museum grounds. Black coefficients are modelled species densities (10m\textsuperscript{2}), whilst red coefficients are the habitat densities that were unable to be modelled and estimated from other habitats (details in Table \ref{tab:wlg1}). Grey coefficients are modelled within-sample species richness and pink coefficients are the within-sample habitat richness of those unable to be estimated. Error bars indicate 95\% confidence intervals.}
   	 \label{fig:wlgmodel}
\end{figure}
	
Calculations for all three assumptions indicate an overall net increase in local biodiversity with the proposed plans for the museum's grounds. Assumption 1 yields an increase of 11.17\%. Under Assumption 2, the increase is estimated to be 13.2\%. Assumption 3 gave the greatest increase (14.05\%) in overall net biodiversity under the proposed plans.

\ifappendixStyle %% thesis is true, journal is false
\subsection{Sensitivity Analysis}%% Thesis
\else
\subsection*{Sensitivity Analysis}
\fi

When the analysis was repeated 1000 times, taking the habitat coefficients from a distribution, the proposed plans under Assumption 1 only resulted in a net loss of biodiversity in 0.4\% of the repetitions (Figure \ref{fig:wlgsensitivity}).

\begin{figure}[t]
	\centering
	\includegraphics[scale=0.5]{SensitivityAnalysis}
	\caption{The number of times each percentage change was obtained in the sensitivity analysis. A random sample was taken from the distribution of each habitat coefficient, and under Assumption 1 the overall biodiversity gain or loss was calculated. This was repeated 1000 times. Vertical line indicates 0\% change. 0.4\% of the runs resulted in a loss of species richness under Assumption 1.}
   	 \label{fig:wlgsensitivity}
\end{figure}

\ifappendixStyle %% thesis is true, journal is false
\subsection{WLG Species Similarity}%% Thesis
\else
\subsection*{WLG Species Similarity}
\fi
The majority of habitats had very similar plant species composition (Figure \ref{fig:wlgsimilarity}), though there were exceptions. For example, very few species from other habitats were found in amenity grass/turf but nearly all species in amenity grass/turf were in most other habitats. Unsurprisingly, ponds had a highly dissimilar collection of species to every other habitat.


\begin{figure}[t]
	\centering
	\includegraphics[scale=0.5]{NHMHabitatsSimiarity}
	\caption{Compositional similarity between habitat types, based on data from the WLG database of plant species. Each cell shows the percentage of species in the habitat on the x-axis that are also present in the habitat listed on the y-axis. Therefore, the grid is not a mirror on the diagonal axis.}
   	 \label{fig:wlgsimilarity}
\end{figure}

\ifappendixStyle %% thesis is true, journal is false
\section{Discussion}%% Thesis
\else
\section*{Discussion}
\fi

The findings of this meta-analysis indicate that the proposed plans for the museum grounds are expected to result in a net gain of local biodiversity. This is because habitats with the highest modelled species density, such as chalk and neutral grassland, will increase in area under the new plans; and because new habitats will be introduced. These findings are similar to those of earlier studies; for instance, a previous synthesis of findings from studies worldwide investigating biodiversity in urban parks found that increasing the habitat area and habitat diversity usually increased species richness \citep{Nielsen:2014ue}. Both area and number of habitats are likely to be important determinants of biodiversity, and potentially it would be more appropriate to incorporate them within a single model (e.g., the Choros model; \citealt{Triantis:2003jb}).

Broadleaved woodlands and neutral grassland contribute greatly to the biodiversity value, and will increase in extent under the proposed plans. As broadleaved woodland will be among the least disturbed habitats during the renovation process, this high-biodiversity-value area may act as a source population for some of the other habitats, especially considering the relatively high proportion of shared plant species in all the other terrestrial habitats.

The statistical methods used in the analysis rely on species diversity modelled as comparisons between habitats. This facilitates comparisons among the fully terrestrial habitats, where sampling methodologies are more likely to be consistent, but, there are unsurprisingly few studies that use a consistent methodology between any fully terrestrial habitat and an aquatic habitat. Thus, the modelled coefficients for ponds are likely to be the least reliable in our analysis, as they are estimated from one comparison against reeds \citep{Williams:2008hb}. However, as ponds make up only a small area of both the current and the proposed grounds (see Figure \ref{fig:wlgplan_after}), this will have little impact on the grounds' overall biodiversity value. Previous work has shown that diversity often increases with pond area \citep{Oertli:2002bc,Parris:2006jae} and pond perimeter \citep{Gagne:2007le,Gagne:2010ec}. With the ponds increasing in area under the proposed plans this may result in a relative increase in species richness compared to the current grounds. However, the proposed plans do have a decrease in the pond's hard edges. Therefore, additional planting of marginal vegetation could increase the perimeter, whilst also increasing diversity \citep{Williams:2008hb,Gioria:2010bc}.


Other limitations and assumptions made in the analysis might also impact the results. In the calculations for the area-scaling of species density, habitat is assumed to be contiguous. Habitats are more contiguous in the plans than in the current grounds, meaning that the proposed plans are relatively disadvantaged in our comparisons. We focused on invertebrates and plants; although these are potentially the most appropriate taxa given the small size of the NHM grounds, it would be ideal to be able to infer the response of vertebrates to the inevitable disturbance.

One of the main objections to the proposed renovation plans voiced by members of the public and other stakeholders is the level of disturbance that will be caused across much of the grounds and the potentially negative impact it will have on biodiversity. We had hoped to be able to model the changes in biodiversity after a disturbance event by modelling how habitat age influences biodiversity. Unfortunately, there were insufficient data for this analysis. In more natural settings, previous work has established that it can take many decades \citep{Hirst:2005jae} -- even a century or more \citep{Vellend:2006ecol} -- for biodiversity to reach levels similar to that before the disturbance. In urban settings, biodiversity is known to increase with habitat age \citep{Yamaguchi:2004er,Sattler:2010le} and age of the surrounding city \citep{Aronson:2014procb}. However, as diversity levels are typically lower than those of natural habitats \citep{Bates:2011po,Ockinger:2009lup}, the time needed to recover could be considerably shorter. Considering that the WLG is only 20 years old, it is unlikely the current biodiversity levels have reached the possible maximum. Even if they are at the peak, the community composition is likely to be different from that of habitats in more natural settings \citep{Angold:2006ste}. 

Post-disturbance natural colonisation may be the main source for biodiversity recovery, and thus an important determinant of dynamics will be the connectivity of the NHM grounds to potential source pools. Many studies have suggested that connectivity within an urban environment is important in maintaining biodiversity \citep{Ockinger:2009lup,Goddard:2010tree,Kong:2010lup,Vergnes:2012bc}. However, an earlier study \citep{Angold:2006ste} reported that landscape variables, such as habitat connectivity, were less important than local site-level variables, such as site age or habitat size, for invertebrate communities in urban environments. This dichotomy of results could be due to the mobility of the studied taxa \citep{Braaker:2014ec}, with highly-mobile species benefiting from connectivity in the landscape more than less-mobile species. We were not able to address this aspect in our study; trait-based statistical models may provide a possible approach \citep[e.g.,][]{Ockinger:2010el,Lizee:2011ei}.

Although the results from this study might not be directly transferable to other case studies, the broad methodology may be suitable in other similar situations. By extending the methods of previous biodiversity models that investigate the impact of land use change on biodiversity, the results can be useful to a single decision-maker for a smaller-scale project \citep[e.g.][]{Newbold:2015nat}. Estimating the impact that disturbance and renovations might ultimately have on biodiversity prior to any undertaking can be valuable, especially when results can directly feed into plans and actions to prevent or offset declines in biodiversity. And while other models exist that assess the potential impact on biodiversity of habitat change and loss \citep[e.g.][]{defra:2012bdo}, meta-analytical methods, such as these, provide empirical and transparent results.

In order to mitigate the impact on biodiversity, the re-development plans involve habitat and tree translocations (alongside native-tree planting and habitat creation). Some habitats within the current WLG were originally translocated from other sites \citep{Honey:1999ln,Leigh:2003ln}, and many habitat translocation schemes have been successful \citep{dunford:2010,Twyford:2012}. However. the approach is not always successful, with some translocations resulting in community changes and reductions of both plants and invertebrates \citep{Bullock:1998bc,jncc:2003ht}. It therefore seems particularly important that the translocated habitats are monitored, with biodiversity targets and objectives set prior to redevelopment. 

The redevelopment of the Natural History's grounds provides the opportunity to monitor aspects of biodiversity recovery within an urban environment that have previously been little studied. Establishing long-term ecological sampling within each of the grounds' habitat types would allow assessment of the recovery of the disturbed habitats as well as the colonisation of the newly created habitats. Standardising the sampling, in conjunction with other projects, would also allow the further comparison of the results with other areas within London \citep[e.g.,][]{Smith:2006ue} or the UK (eg. BUGS2 project: \citealt{Loram:2007le}). Long term regular sampling would also provide the opportunity for other hypotheses to be rigorously tested, for example, whether reduced connectivity in urban areas increases genetic differentiation between populations \citep{Johnson:2015ajb}. Of course, with the grounds being an integral part of the Natural History Museum they provide unique opportunities for the monitoring to be undertaken not only by the taxon experts on Museum staff but also by members of the public \citep{Silvertown:2009tree,Roy:2012citsci}, not only reducing costs but also increasing public engagement in and participation in science and awareness of the new grounds and urban biodiversity in general. 

Urban green areas face many threats worldwide. Without robust methods for estimating the consequences for biodiversity, planning decisions could be uninformed or misinformed. We show how a global modelling approach can be downscaled to inform local decision-makers about the impact of habitat change on species richness and species density. In the case of the Natural History Museum, London, the proposed changes to the grounds are predicted to result in a net gain of biodiversity, due to increases in the number and areas of habitat types. The size of the gain depends on the assumptions made about the relationship between within-sample species diversity and habitat area -- a relationship that has been largely ignored in global models. The grounds redevelopment program provides an opportunity for systematic ecological surveys to quantify the effects of habitat creation, expansion, reduction and disturbance, adding useful knowledge about this culturally important urban green space at the same time as allowing improvement of biodiversity models to support planning decisions.

